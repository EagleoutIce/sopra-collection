\documentclass{sopra-base}

\usepackage{sopra-documentation}

\title{Die 'sopra-base'-Klasse}
\subtitle[Dokumentation für die 'sopra-base'-Klasse]{Dokumentation für die 'sopra-base'-Klasse | Version \thesobversion}

\duedate{2019-12-2}

\keywords{Dokumentation,sopra-base,sopra,uni ulm,uulm}
\authors{Florian Sihler (florian.sihler@uni-ulm.de)}

\group{Die Affenbande}
\begin{document}
    \maketitle%
%
    %\tableofcontents
%

\section{Allgemeines}
\subsection{Warum, wieso, weshalb?}
    Diese \LaTeXe-Dokumentlasse wurde im Rahmen des Sopras im
    Wintersemester 2019 und Sommersemester 2020 verfasst und dient als
    Grundlage für die Optik aller Dokumente des \imptext{Teams 20} und kann
    binnen seiner Entwicklung noch grundlegend verändert werden. Das
    bisherige Design ist an das Aufgabentemplate der Universität Ulm, wobei
    eine Schnittstelle geschaffen wurde, die auch bei varrierenden optischen
    Eigenschaftendes Dokuments keine Änderungen im \LaTeX-Codes erfordern
    sollten.\par
    Zum Visualisieren der einzelnen Code-Ausschnitte wird das
    \T{sopra-listings}-Paket verwendet. Es ist für die Verwendung der Dokumentklasse
    nicht relevant.\par
    Die zugehörige Klasse sollte ebenfalls in dieses Dokument eingebettet sein: \scalebox{0.65}{\attachfile[subject={sopra-base.cls}]{sopra-base.cls}}.
\subsection{Abhängigkeiten}
    Dieses Paket basiert auf der, jeder \LaTeX-Distribution beigelegten,
    \T{article}-Klasse,\footnote{\url{https://ctan.org/pkg/article}}
    sowie den folgenden Paketen (alle übergebenen Argumente werden ebenfalls
    angegeben, wobei im Fall einer Werteübergabe ein Stern gesetzt wird.):
    \begin{multicols}{3}
        \begin{itemize}
            \def\pkgparse#1:#2\@nil{%
                \T{#1}\ifx!#2!\else\textsuperscript{[#2]}\fi%
            }
            \foreach \pkg in {tikz:,fontenc:T1, inputenc:utf8,
                babel:{english,main=ngerman},geometry:{margin*,
                a4paper,bottom*},sfmath:,lmodern:,microtype:,
                scrlayer-scrpage:,titlesec:,enumitem:,nowidow:all,hyperref:hidelinks}{
                \item \expandafter\pkgparse\pkg\@nil
            }
        \end{itemize}
    \end{multicols}
    All diese Pakete sollten Teil der gängigen \LaTeX-Distribution sein (weiter
    wird noch die Ti\textit{k}Z-Bibliothek \T{calc} verwendet, sowie \T{shapes}
    wenn eine Grupennummer mittels \cmdref{groupnum} oder \cmdref{setTeam}
    angegeben wird.)
\subsection{Die Installation}
    Die Klasse wird nicht als \T{.dtx} ausgeliefert, weswegen sich die
    folgenden Möglichkeiten ergeben:
    \begin{itemize}
        \item Die Dokumentklasse kann in dasselbe Verzeichnis wie das Dokument
                gesetzt werden. In diesem Fall lautet die Einbindungsanweisung:
\begin{plainlatex}
\documentclass{sopra-base}
\end{plainlatex}
        \item Die Dokumentklasse kann in ein Unterverzeichnis/in ein mit
                dem Dokument ausgeliefertes Verzeichnis gelegt werden. In
                diesem Fall erfolgt die Angabe durch den (relativen-) Pfad:
\begin{plainlatex}
\documentclass{./Mein/Pfad/zu/sopra-base}
\end{plainlatex}
        \item Man kann die Klasse (mittels eines Symlinks oder ähnlichem)
              in einen eigenen \emph{texmf}-Baum ablegen.
              So kann zum Beispiel auf Linux unter der Verwendung von texlive
              die Klasse hier abgelegt werden: \bvoid{\~/texmf/tex/latex/}.
              Das Verzeichnis kann erstellt und anschließend mittels
              \bbash{texhash \~/texmf} aktualisiert werden. Nun kann
              die Klasse wie jede andere installierte Klasse verwendet werden:
\begin{plainlatex}
\documentclass{sopra-base}
\end{plainlatex}
    \end{itemize}
    \subsection{Weitere Besonderheiten}
    In Version \thesobversion{} (\cmdref{thesobversion}) gibt es keine weiteren
    Besonderheiten.

\section{Klassen-Konfiguration}
    \subsection{Akzeptierte Parameter}
    Die Dokumentklasse akzeptiert, so wie die meisten, Argumente. So können
    nebst den für \T{article} akzeptierten Argumente die im folgenden
    aufgelistet werden, wobei bei Argumenten mit einer \say{Counter}-Option
    das jeweils standardmäßig aktive zuerst und das andere in Klammern
    geschrieben. So wird implizit:
\begin{plainlatex}
    \documentclass[final,nopar,sffont,defaultmode,showmail]{sopra-base}
\end{plainlatex}
    aufgerufen. Während wir mit:
\begin{plainlatex}
    \documentclass[rmfont]{sopra-base}
\end{plainlatex}
    das Dokument in der altbekannten \emph{Roman}-Schrift setzen lassen können.

    \begin{argument}{final}{draft}
        Im \T{draft}-Modus können einige Elemente so wie Grafiken nicht
        gesetzt werden um so den Kompilierprozess zu beschleunigen.
        Diese Option wird analog auch von vielen weiteren Paketen akzeptiert.
    \end{argument}

    \begin{argument}{nopar}{dopar}
        Mittels \T{dopar} setzen wir die Einrückung bei einem Paragraphen
        auf den von \LaTeX-gesetzten Standartwert.
    \end{argument}


    \begin{argument}{sffont}{rmfont}
        Durch \T{rmfont} wird die Schriftart wieder auf \emph{Roman} gesetzt.
    \end{argument}

    \begin{argument}{defaultmode}{print}
        Mittels \T{print} werden die Farben verringt beziehungsweise in Teilen
        komplett entfernt (in Graustufen konvertiert) um ein besseres Ergebnis
        im Druck zu erzielen.
    \end{argument}

    \begin{argument}{showmail}{hidemail}
        Mit \T{hidemail} wird die E-Mail-Adresse bei der Angabe der Autoren
        nicht angezeigt.
    \end{argument}

    \subsection{Weitere Stile}
    In der aktuellen Version (\thesobversion) werden noch keine alternativen
    Stile und Varianten unterstützt. Sie werden vermutlich über weitere
    Parameter oder Subklassen realisiert werden.

\section{Befehlsübersicht}

Die Klasse fügt als \emph{base}-Klasse keinen großen Satz an Befehlen für
den Nutzer hinzu. Diese Aufgabe gebührt den erweiternden Klassen und Paketen.

\subsection{Daten setzen}
\label{sec:DatenSetzen}
Die Folgenden Befehle sollten in der Präambel gesetzt werden und konfigurieren
auch die Metadaten des jeweiligen Dokuments (also die Dokumenteigenschaften).
Sie können beliebig oft überschrieben werden (bis zu dem Punkt, an dem
das Dokument beginnt). \notetext{Entwicklernotiz: Alle diese Felder stehen über
\T{\textbackslash sob@register@$\langle$Name$\rangle$}, beziehungsweise ihre 'short'-Varianten
über \T{\textbackslash sob@register@short@$\langle$Name$\rangle$} zur Verfügung.}

\begin{command}{title}{\optArg{short}\manArg{NeuerTitel}}
    Setzt den Titel des Dokuments auf \T{NeuerTitel}.
\end{command}

\begin{command}{subtitle}{\optArg{short}\manArg{NeuerTitel}}
    Setzt den Untertitel des Dokuments auf \T{NeuerTitel}.
\end{command}

\begin{command}{brief}{\optArg{short}\manArg{Kurzbeschreibung}}
    Setzt die Kurzbeschreibung des Dokuments auf \T{Kurzbeschreibung}.
    Dieses Datenfeld wird in der Basis-Klasse nicht verwendet.
\end{command}

\begin{command}{authors}{\optArg{short}\manArg{Autorenliste}}
    Hierrüber kann die Liste der Autoren gesetzt werden. Es empfiehlt sich
    hierfür \cmdref{addAuthor} zu verwenden.
\end{command}

\begin{command}{duedate}{\optArg{short}\manArg{YYYY-MM-DD}}
   Setzt das Datum des Dokuments. Zur Konvertierung wird \cmdref{DateConvert}
   verwendet.
\end{command}

\begin{command}{supervisor}{\optArg{short}\manArg{Name}}
    Setzt den Betreuer auf \T{Name}. Dieses Datenfeld wird in der Basis-Klasse nicht verwendet.
\end{command}

\begin{command}{keywords}{\optArg{short}\manArg{Schlüsselwörter}}
    Setzt die Keywords für die Metadaten auf \T{Schlüselwörter}.
\end{command}

\begin{command}{group}{\optArg{short}\manArg{Name}}
    Setzt die Gruppe des jeweiligen Erstellers auf \T{Name}.
\end{command}

\begin{command}{groupnum}{\optArg{short}\manArg{Nummer}}
    Setzt im Falle einer Nummerierung (oder Symbol, etc.) der Gruppe dieses
    auf \T{Nummer}.
\end{command}

\subsection{Beim Datensetzen hilfreiches}
\begin{command}{addAuthor}{\manArg{Name (Email)}}
    Fügt einen Autor der Liste der Autoren (\cmdref{authors}) hinzu. So zum
    Beispiel: \blatex[morekeywords={[5]{\\addAuthor}}]{\\addAuthor\{Florian Sihler (florian.sihler@uni-ulm.de)\}}.
\end{command}

\begin{command}{setTeam}{\manArg{Name Nummer}}
    Erlaubt es \cmdref{group} und \cmdref{groupnum} in einem Aufruf zu setzen.
    So zum Beispiel: \T{\cmd{setTeam}\{Affenbande 42\}} oder
    \T{\cmd{setTeam}\{\{Mega Team\} A\}}.
\end{command}

\subsection{Die Daten anzeigen}

\begin{command}{typesetAuthors}{}
    Setzt die Autoren gemäß \jmark[\T{showmail}]{mrk:showmail} mit oder
    ohne E-Mail-Adresse, so ergibt \cmd{typesetAuthors}: \typesetAuthors.
    Bei mehreren werden Kommas und das \say{und} korrekt gesetzt.
\end{command}

\begin{command}{maketitle}{}
    Setzt den Titel auf Basis der \jmark[gesetzten Daten]{sec:DatenSetzen}
\end{command}

\begin{command}{thesobversion}{}
    Gibt die aktuelle Version der \T{sopra-base}-Dokumentklasse aus. \notetext{Hinweis: über \blatex{\\value\{sobversion\}} lässt sich
    die Version als $4$-stellige Nummer erhalten: \arabic{sobversion}.}
\end{command}

\begin{command}{DateConvert}{\manArg{YYYY-MM-DD}}
    Konvertiert ein Datum in eine schönere Form, so:
\begin{plainlatex}[morekeywords={[5]{\\DateConvert}}]
\DateConvert{2019-12-02} % :yields: !*\solGet{comments}{\DateConvert{2019-12-02}}*!
\end{plainlatex}
\end{command}

\subsection{Generelle Formatierungsbefehle}

\begin{command}{imptext}{\manArg{Text}}
    Hebt \T{Text} als \imptext{sehr wichtig} hervor.
\end{command}

\begin{command}{notetext}{\manArg{Text}}
    Setzt \T{Text} als eine \notetext{beiläufige Notiz}.
\end{command}

\appendix
\section{Verbundene Klassen und Pakete}

Im Zuge dieser \LaTeXe-Dokumentklasse sind noch einige andere Pakete entstanden. Zum aktuellen Kompilierstand existiert (Pakete und Klassen, ohne Link sind dieser beigelegt und hier: \url{https://github.com/EagleoutIce/sopra-collection} zu finden):
\begin{multicols}{3}
    \begin{itemize}
        \foreach \i/\u in {eagle-maps/{https://github.com/EagleoutIce/eagle-maps},LILLY-Framework/{https://github.com/EagleoutIce/LILLY},sopra-documentation/,sopra-models/,sopra-listings/,sopra-requirements/,sopra-attachments/} {
            \item \T{\i}\expandafter\ifx\expandafter\\\u\\\else\footnote{\url{\u}}\fi
        }
    \end{itemize}
\end{multicols}

\end{document}