\documentclass{sopra-base}

\usepackage{sopra-documentation}
\usepackage{sopra-tables}

\title{Das 'sopra-tables'-Paket}
\subtitle[Dokumentation für das 'sopra-tables'-Paket]{Dokumentation für das 'sopra-tables'-Paket | Version \thesotversion}
\duedate{2019-12-5}

\keywords{Dokumentation,sopra-tables,sopra,uni ulm,uulm,Paket}
\authors{Florian Sihler (florian.sihler@uni-ulm.de)}

\group{Die Affenbande}
\begin{document}

    \maketitle%
%
    %\tableofcontents
%

%
%
% ....###....##.......##........######...########.##.....##.########.####.##....##.########..######.
% ...##.##...##.......##.......##....##..##.......###...###.##........##..###...##.##.......##....##
% ..##...##..##.......##.......##........##.......####.####.##........##..####..##.##.......##......
% .##.....##.##.......##.......##...####.######...##.###.##.######....##..##.##.##.######....######.
% .#########.##.......##.......##....##..##.......##.....##.##........##..##..####.##.............##
% .##.....##.##.......##.......##....##..##.......##.....##.##........##..##...###.##.......##....##
% .##.....##.########.########..######...########.##.....##.########.####.##....##.########..######.
%
%

\section{Allgemeines}
\subsection{Warum, wieso, weshalb?}
    Dieses \LaTeXe-Paket wurde im Rahmen des Sopras im
    Wintersemester 2019 und Sommersemester 2020 verfasst und dient als
    Grundlage für die Erstellung von Tabellen
    des \imptext{Teams 20}. Diese Dokumentation wurde zusammen mit der
    \T{sopra-base.cls} sowie dem Paket \T{sopra-documentation.sty} kreiert.\par
    Zum Visualisieren der einzelnen Code-Ausschnitte wird das
    \T{sopra-listings}-Paket verwendet.
    Das zugehörige Paket sollte ebenfalls in dieses Dokument eingebettet sein: \scalebox{0.65}{\attachfile[subject={sopra-tables.sty}]{sopra-tables.sty}}.
\subsection{Abhängigkeiten}
    Dieses Paket bindet die folgenden Paketen mit ein:
    \begin{multicols}{3}
        \begin{itemize}
            \def\pkgparse#1:#2\@nil{%
                \T{#1}\ifx\\#2\\\else\textsuperscript{(#2)}\fi%
            }
            \foreach \pkg in {tabu:{\jmark[\T{usetabu}]{mrk:usetabu}},colortbl:{\jmark[\T{usetabu}]{mrk:usetabu}},booktabs:{\jmark[\T{notabu}]{mrk:usetabu}},makecell:,longtable:,amssymb:,xcolor:} {
                \item \expandafter\pkgparse\pkg\@nil
            }
        \end{itemize}
    \end{multicols}
    All diese Pakete sollten Teil der gängigen \LaTeX-Distribution sein.

\subsection{Die Installation}
    Das Paket wird nicht als \T{.dtx} ausgeliefert, weswegen sich die
    folgenden Möglichkeiten ergeben:
    \begin{itemize}
        \item Das Paket kann in dasselbe Verzeichnis wie das Dokument
                gesetzt werden. In diesem Fall lautet die Einbindungsanweisung:
\begin{plainlatex}
\usepackage{sopra-tables}
\end{plainlatex}
        \item Das Paket kann in ein Unterverzeichnis/in ein mit
                dem Dokument ausgeliefertes Verzeichnis gelegt werden. In
                diesem Fall erfolgt die Angabe durch den (relativen-) Pfad:
\begin{plainlatex}
\usepackage{./Mein/Pfad/zu/sopra-tables}
\end{plainlatex}
        \item Man kann das Paket (mittels eines Symlinks oder ähnlichem)
              in einen eigenen \emph{texmf}-Baum ablegen.
              So kann zum Beispiel auf Linux unter der Verwendung von texlive
              das Paket hier abgelegt werden: \bvoid{\~/texmf/tex/latex/}.
              Das Verzeichnis kann erstellt und anschließend mittels
              \bbash{texhash \~/texmf} aktualisiert werden. Nun kann
              das Paket wie jede andere installierte Paket verwendet werden:
\begin{plainlatex}
\usepackage{sopra-tables}
\end{plainlatex}
    \end{itemize}

\subsection{Weitere Besonderheiten}
In Version \thesotversion{} (\cmdref{thesotversion}) gibt es keine weiteren
Besonderheiten.

%
%
% .##....##..#######..##....##.########.####..######...##.....##.########.....###....########.####..#######..##....##
% .##...##..##.....##.###...##.##........##..##....##..##.....##.##.....##...##.##......##.....##..##.....##.###...##
% .##..##...##.....##.####..##.##........##..##........##.....##.##.....##..##...##.....##.....##..##.....##.####..##
% .#####....##.....##.##.##.##.######....##..##...####.##.....##.########..##.....##....##.....##..##.....##.##.##.##
% .##..##...##.....##.##..####.##........##..##....##..##.....##.##...##...#########....##.....##..##.....##.##..####
% .##...##..##.....##.##...###.##........##..##....##..##.....##.##....##..##.....##....##.....##..##.....##.##...###
% .##....##..#######..##....##.##.......####..######....#######..##.....##.##.....##....##....####..#######..##....##
%
%

\subsection{Akzeptierte Parameter}
    Das Paket akzeptiert, so wie die meisten, Argumente.
    Bei Argumenten mit einer \say{Counter}-Option wird das jeweils standardmäßig aktive zuerst und das andere in Klammern
    geschrieben. So wird implizit:
\begin{plainlatex}
    \usepackage[usetabu]{sopra-tables}
\end{plainlatex}
    aufgerufen. Während wir mit:
\begin{plainlatex}
    \usepackage[notabu]{sopra-tables}
\end{plainlatex}
    das Dokument ohne das \T{tabu}-Paket kompilieren.

    \begin{argument}{usetabu}{notabu}
        Analog zur \T{notabu} option der \T{sopra-requirements.sty}: Standardmäßig verwendet dieses Paket aus optischen Gründen \T{tabu} um die Tabellen zu visualisieren. Sollten die in diesem paket enthaltenen (bug-)fixes (nicht mehr) funktionieren, oder einem die in \T{notabu} verwendeten \T{booktabs} Tabellen einem bessere gefallen, so kann diese Option entsprechend umgestellt werden.  \notetext{Hinweis: Mit \T{usetabu} wird innerhalb von \env{mtabular} auf \cmd{relax} gesetzt, damit kein Fehler/keine horizontalen Linien entstehen, wenn man für beide designs schreiben möchte.}
    \end{argument}

    \begin{argument}{cpalette}{nocpalette}
        Bietet Unterstützung für \href{color-palettes}{https://github.com/EagleoutIce/color-palettes}. In diesem Fall wird das highlighting der Tabelle automatisch an die aktuelle Palette angepasst.
    \end{argument}

    \begin{argument}{invert}{noinvert}
        Invertiert die Hervorhebungen für die Kopfzeile.
    \end{argument}

    \subsection{Farben}

    Mit \jmark[\T{usetabu}]{mrk:usetabu} definiert die Farbe \T{MaterialHeaderColor} die Farbe der Titelspalte und \T{NextMaterialHeaderColor} die Farbe im Falle eines Seitenumbruchs. Die Farben können wie jede andere geändert werden. Innerhalb von \T{sopra-base} wird automatisch eine dem Design angepasste Farbe gewählt.

%
%
% .########..########.########.########.##.....##.##.......########
% .##.....##.##.......##.......##.......##.....##.##.......##......
% .##.....##.##.......##.......##.......##.....##.##.......##......
% .########..######...######...######...#########.##.......######..
% .##.....##.##.......##.......##.......##.....##.##.......##......
% .##.....##.##.......##.......##.......##.....##.##.......##......
% .########..########.##.......########.##.....##.########.########
%
%

\section{Befehle- und Umgebungen}

Es gilt zu beachten, dass das Präfix \T{env@} nur auf die Natur einer Umgebung hinweist und nicht zum eigentlichen Bezeichner zuzuordnen ist!

\subsection{Die Tabellen}

\begin{environment}{mtabular}{\manArg{Spaltendefinition}}
    Setzt eine \T{tabu}/\T{tabular} Tabelle, wobei die erste Zeile automatisch zur Titelzeile wird und entsprechend farblich hervorgehoben.
\end{environment}

\begin{environment}{mltabular}{\optArg{Header}\manArg{Spaltendefinition}}
    Setzt eine \T{longtabu}/\T{longtable} Tabelle, wobei man durch Angabe des \T{Header} die Kopfspalte definiert, die im Falle eines Seitenumbruchs immer angefügt wird. Sollte dies nicht gewünscht sein, so genügt es den \T{Header} nicht anzugeben und ganz normal die erste Zeile als Titelzeile zu schreiben,
    die nicht an jeden Seitenanfang gesetzt werden. Es gelten die von \T{longtable} bekannten und typischen Einschränkungen.
\end{environment}

\subsection{Allgemeine Hinweise und Tipps}

Damit ein \T{longtabu}/\T{longtable} korrekt angezeigt wird, wird in der Regel ein zweiter Kompiliervorgang benötigt.\par{}
Es empfiehlt sich die \env{table}-Umgebung zusammen mit \cmd{caption} zu verwenden umd die Tabelle sinnvoll einzubetten. Es kommt auch \env{wrapfigure} in Frage.\par{}
Wird auch das \T{sopra-requirements}-Paket eingebunden, können für zweispaltige Tabellen die Spaltentypen \T{K} und \T{V} verwendet werden.

\subsection{Beispiel}

Im Folgenden seien zwei Tabellen veranschaulicht, es gilt zu beachten, dass \envref{mltabular} seines Ursprungs wegen automatisch zentriert vorliegt.
\begin{latex}
\begin{mtabular}{lcr}
    Hallo & Welt & \texttt{Wie} gehts? \\
    Wie stehts & denn heute & so dir und so? \\
    Bi & Bu & bo \\
    Hi & Hi & Hi \\
\end{mtabular}
\end{latex}
\textit{Ergibt:}\par{}
 \begin{mtabular}{lcr}
    Hallo & Welt & \texttt{Wie} gehts? \\
    Wie stehts & denn heute & so dir und so? \\
    Hi & Hi & Hi \\
    Bi & Bu & bo \\
\end{mtabular}\par{}
Und analog:\par{}
\begin{latex}
\begin{mltabular}[Isch & bin & a Hädder! & Epic-Row]%
        {lcr>{\raggedleft!**!\arraybackslash}m{5cm}}
    Hallo & Welt & \texttt{Wie} gehts? & HUHU ERDE \\
    Wie stehts & denn heute & so dir und so? &  Hi\\
    Bi & Bu & bo & Man merkt ich bin \\
    Suuper & Lange Tabelle & Weil Grund  & echt Kreativ\\
    A1 & B8 & C1 & D8 \\
    A2 & B7 & C2 & D7 \\
    A3 & B6 & C3 & D6 \\
    A4 & B5 & C4 & D5 \\
    A5 & B4 & C5 & D4 \\
    A6 & B3 & C6 & D3 \\
    A7 & B2 & C7 & D2 \\
    A8 & B1 & C8 & D1 \\
\end{mltabular}
\end{latex}
\textit{Ergibt \notetext{Hinweis: der extra Abstand hier ist künstlich und dient der Veranschualichung eines Seitenumbruchs}:}\vspace*{4cm}
\begin{mltabular}[Isch & bin & a Hädder! & Epic-Row]{lcr>{\raggedleft\arraybackslash}m{5cm}}
    Hallo & Welt & \texttt{Wie} gehts? & HUHU ERDE \\
    Wie stehts & denn heute & so dir und so? & Hi\\
    Bi & Bu & bo & Man merkt ich bin \\
    Suuper & Lange Tabelle & Weil Grund  & echt Kreativ\\
    A1 & B8 & C1 & D8 \\
    A2 & B7 & C2 & D7 \\
    A3 & B6 & C3 & D6 \\
    A4 & B5 & C4 & D5 \\
    A5 & B4 & C5 & D4 \\
    A6 & B3 & C6 & D3 \\
    A7 & B2 & C7 & D2 \\
    A8 & B1 & C8 & D1 \\
\end{mltabular}
\end{document}