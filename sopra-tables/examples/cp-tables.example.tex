\documentclass{article}

\usepackage{hyperref}
\usepackage[hyperref,addons]{../../color-palettes/color-palettes}
\usepackage[cpalette]{../sopra-tables}
\usepackage{pgffor}
\solLoadLanguage{xml,json}

\title{Tables-Beispiel}
\author{Florian Sihler}
\date{\today}

\makeatletter%
\def\doinvert{%
    \sot@invert@true%
    \let\sot@table@headerstyle\sot@table@headerstyle@inv%
    \sot@init%
}
\makeatother%

\begin{document}
\maketitle

\tableofcontents
\clearpage
\edef\AP{\AllPalettes}
\foreach \mode/\c in {Normal/{},Invertiert/\doinvert} {
\section{Die Auswirkungen der Paletten (\mode)}
\c%
\foreach \i in \AP {\UsePalette{\i}%
% to assure the breaks :D
\subsection{Palette: \CurrentPalette}%
\begin{center}
    \ShowcaseCurrentPalette%
\end{center}\medskip
\begin{mtabular}{lcr}
    Kleine & Feine & Tabelle \\
    Ja was & denkt man & auf die \\
    schnelle & denn vom & Gefährten \\
    der da & wandelt. & \\
    Er strandet gar & im Glücke war & und lebte \\
    dar was & ewig war & und offen bar \\
    Jap & dap & jeah.
\end{mtabular}

\begin{mltabular}[Ich bin & ein langer & Tabular]{lcr}
    Hipp & Hipp & Jaha \\
    Ja ich & bin ein & tabular \\
    und ein & langer offen & bar. \\
    Ganz zentriert, & dass bin ich & Tscha-ka\\
    & Ich brauch noch & ein zwei Zeilen ja \\
    und schon & ist der & Seitenumbruch \\
    da. & Hoffe ich & denn irgendwie \\
    schreibe ich die & Zeilen nie & ohne einen Plan. \\
    & Wer das & liest \\
    der & sei gewarnt & \\
    verzählten & & Zeilen stählen \\
    Unsinn er- & zählen & \\
    Tra- & la- & la. \\
    Neue Seite & Sonna klar  & \\
    Ih mag net mehr & soh'n schwachsinn & mach-a \\
    Ha & Ha & Ha
\end{mltabular}\clearpage
}
}
\end{document}