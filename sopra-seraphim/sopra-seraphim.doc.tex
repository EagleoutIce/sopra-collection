\documentclass{sopra-seraphim}

\usepackage{sopra-documentation}

\makeatletter
% #1: name of the present-sequence
% #2: Args of the present-sequence
% #3: presenter style
% #4: prefix of cs-name
\renewenvironment{present}[4]{\phantomsection\label{mrk:#1}%
    \begingroup\leavevmode\newline\noindent{\quad}{#3{#4#1}}{\T{#2}}\par\nopagebreak% done on purpose
}{\endgroup\bigskip}
\makeatother

\usepackage{sopra-attachments}

\title{Die 'sopra-seraphim'-Klasse}
\subtitle{Dokumentation für die 'sopra-seraphim'-Klasse, Version \thesosversion}

\duedate{2019-12-15}

\keywords{Dokumentation,sopra-seraphim,sopra,uni ulm,uulm}
\authors{Florian Sihler (florian.sihler@uni-ulm.de)}

% \group{Die Affenbande}
\groupnum{020}

\begin{document}
    \Intro

    \section{Die 'sopra-seraphim'-Klasse}
    \subsection{Warum, wieso, weshalb?}
    \begin{frame}{Warum 'sopra-seraphim'?}
        Diese \LaTeXe-Dokumentlasse wurde im Rahmen des Sopras im
        Wintersemester 2019 und Sommersemester 2020 verfasst und dient als
        Grundlage für die Optik aller Dokumente des \notetext{Teams 20}.
        Das Layout basiert auf \T{beamer}. Die Dokumentation wurde aus strukturellen Gründen in letzterem angefertigt, es werden also keine Titel platziert, analog passt sich die Kopfzeile an den Umstand an!\par
        Zum Visualisieren der einzelnen Code-Ausschnitte wird das
        \T{sopra-listings}-Paket verwendet. Es ist für die Verwendung der Dokumentklasse
        nicht relevant.\par
        Die zugehörige Klasse sollte ebenfalls in dieses Dokument eingebettet sein: \textattachfile[subject={sopra-seraphim.cls}]{sopra-seraphim.cls}{sopra-seraphim.cls}.
    \end{frame}

    \subsection{Abhängigkeiten}
    \begin{frame}[fragile]{Abhängigkeiten}
    Dieses Paket basiert auf \T{beamer}, sowie den folgenden Paketen (alle übergebenen Argumente werden ebenfalls
    angegeben, wobei im Fall einer Werteübergabe ein Stern gesetzt wird. Ein Ausrufezeichen vor einem \say{Argument} deklariert, dass es sich hier um eine Option handelt die das Laden des Pakets kontrolliert):
    \begingroup\footnotesize
    \begin{multicols}{3}
        \begin{itemize}
            \def\pkgparse#1:#2\@nil{%
                \T{#1}\ifx!#2!\else\textsuperscript{(#2)}\fi%
            }
            \foreach \pkg in {tikz:,fontenc:T1, inputenc:utf8,
                babel:{english,main=ngerman},montserrat:{defaultfam,*},microtype:,nowidow:all,hyperref:hidelinks}{
                \item \expandafter\pkgparse\pkg\@nil
            }
        \end{itemize}
    \end{multicols}
    \endgroup
    All diese Pakete sollten Teil der gängigen \LaTeX-Distribution sein (weiter
    wird noch die Ti\textit{k}Z-Bibliothek \T{calc} verwendet.
    \end{frame}


    \subsection{Die Installation}
    \begin{frame}[fragile]{Installation: Pfadangabe}
        Die Klasse wird nicht als \T{.dtx} ausgeliefert, weswegen sich die
    folgenden Möglichkeiten ergeben:
\begin{itemize}
    \item Die Dokumentklasse kann in dasselbe Verzeichnis wie das Dokument
    gesetzt werden. In diesem Fall lautet die Einbindungsanweisung:
\begin{plainlatex}
\documentclass{sopra-seraphim}
\end{plainlatex}
    \item Die Dokumentklasse kann in ein Unterverzeichnis/in ein mit
                dem Dokument ausgeliefertes Verzeichnis gelegt werden. In
                diesem Fall erfolgt die Angabe durch den (relativen-) Pfad:
\begin{plainlatex}
\documentclass{./Mein/Pfad/zu/sopra-seraphim}
\end{plainlatex}
\end{itemize}

    \end{frame}

    \begin{frame}[fragile]{Installation: Im texmf-Baum}
        Man kann die Klasse (mittels eines Symlinks oder ähnlichem)
        in einen eigenen \emph{texmf}-Baum ablegen.
        So kann zum Beispiel auf Linux unter der Verwendung von texlive
        die Klasse hier abgelegt werden: \bvoid{\~/texmf/tex/latex/}.
        Das Verzeichnis kann erstellt und anschließend mittels
        \bbash{texhash \~/texmf} aktualisiert werden. Nun kann
        die Klasse wie jede andere installierte Klasse verwendet werden:
\begin{plainlatex}
\documentclass{sopra-seraphim}
\end{plainlatex}
        \notetext{Hinweis: Hierfür existiert ein Python3.5-Skript namens \textattachfile{installer.py}{installer.py} welches, im darüberliegenden Ordner plaziert, die Installation/Aktualisierung übernehmen kann.}
    \end{frame}

    \subsection{Weitere Besonderheiten}
    \begin{frame}[fragile]{Weitere Besonderheiten}
        In Version \thesosversion{} (\cmdref{thesosversion}) gibt es keine weiteren Besonderheiten.
    \end{frame}



    \section{Klassen-Konfiguration}
    \subsection{Akzeptierte Parameter}
    \begin{frame}[fragile]{Allgemeines zu Parametern}
    Die Dokumentklasse akzeptiert, so wie die meisten, Argumente. So können
    nebst den für \T{article} akzeptierten Argumente die im folgenden
    aufgelistet werden, wobei bei Argumenten mit einer \say{Counter}-Option
    das jeweils standardmäßig aktive zuerst und das andere in Klammern
    geschrieben. So wird implizit:
\begin{plainlatex}
\documentclass[setstyle]{sopra-seraphim}
\end{plainlatex}
    aufgerufen. Während dieses Dokument mit:
\begin{plainlatex}
\documentclass[nosetstyle]{sopra-seraphim}
\end{plainlatex}
    der Stil anderer \T{sopra-*} Pakete nicht modifiziert wird.
    \end{frame}

    \begin{frame}[fragile]{Die Parameter}
        \begin{argument}{setstyle}{nosetstyle}
            Mit \T{setstyle} wird automatisch geprüft, welche Pakete der \T{sopra-collection} (\url{https://github.com/EagleoutIce/sopra-collection}) geladen sind und entsprechende Stile gesetzt um ein einheitliches Layout zu erzeugen.
        \end{argument}
    \end{frame}


    \section{Befehlsübersicht}
    \begin{frame}{Befehlsübersicht}
        Die Klasse fügt als Basis-Klasse keinen großen Satz an Befehlen für
        den Nutzer hinzu. Diese Aufgabe gebührt den erweiternden Paketen.
    \end{frame}


    \subsection{Daten setzen}
    \label{sec:DatenSetzen}
    \begin{frame}[fragile,allowframebreaks]{Daten setzen}
        Die Folgenden Befehle sollten in der Präambel gesetzt werden und konfigurieren
        auch die Metadaten des jeweiligen Dokuments (also die Dokumenteigenschaften).
        Sie können beliebig oft überschrieben werden (bis zu dem Punkt, an dem
        das Dokument beginnt). \notetext{Entwicklernotiz: Alle diese Felder stehen über
        \T{\textbackslash sos@register@$\langle$Name$\rangle$}, beziehungsweise ihre 'short'-Varianten
        über \T{\textbackslash sos@register@short@$\langle$Name$\rangle$} zur Verfügung.}

        \begin{command}{brief}{\optArg{short}\manArg{Kurzbeschreibung}}
            Setzt die Kurzbeschreibung des Dokuments auf \T{Kurzbeschreibung}. Sie wird auf der Titelseite links unten angezeigt. \imptext{Wird aktuell nicht verwendet.}
        \end{command}

        \begin{command}{authors}{\optArg{short}\manArg{Autorenliste}}
            Hierrüber kann die Liste der Autoren gesetzt werden. Es empfiehlt sich
            hierfür \cmdref{addAuthor} zu verwenden.
        \end{command}

        \begin{command}{duedate}{\optArg{short}\manArg{YYYY-MM-DD}}
           Setzt das Datum des Dokuments. Zur Konvertierung wird \cmdref{DateConvert}
           verwendet.
        \end{command}

        \begin{command}{supervisor}{\optArg{short}\manArg{Name}}
            Setzt den Betreuer auf \T{Name}. Dieses Datenfeld wird in der Basis-Klasse nicht verwendet. \imptext{Wird aktuell nicht verwendet.}
        \end{command}

        \begin{command}{keywords}{\optArg{short}\manArg{Schlüsselwörter}}
            Setzt die Keywords für die Metadaten auf \T{Schlüselwörter}.
        \end{command}

        \begin{command}{group}{\optArg{short}\manArg{Name}}
            Setzt die Gruppe des jeweiligen Erstellers auf \T{Name}.% Maybe show?
        \end{command}

        \begin{command}{groupnum}{\optArg{short}\manArg{Nummer}}
            Setzt im Falle einer Nummerierung (oder Symbol, etc.) der Gruppe dieses
            auf \T{Nummer}.
        \end{command}
    \end{frame}


    \subsection{Beim Datensetzen hilfreiches}
    \begin{frame}[fragile]{Beim Datensetzen hilfreiches}
    \begin{command}{addAuthor}{\manArg{Name (Email)}}
        Fügt einen Autor der Liste der Autoren (\cmdref{authors}) hinzu. So zum
        Beispiel: \blatex[morekeywords={[5]{\\addAuthor}}]{\\addAuthor\{Florian Sihler (florian.sihler@uni-ulm.de)\}}.
    \end{command}

    \begin{command}{setTeam}{\manArg{Name Nummer}}
        Erlaubt es \cmdref{group} und \cmdref{groupnum} in einem Aufruf zu setzen.
        So zum Beispiel: \T{\cmd{setTeam}\{Affenbande 42\}} oder
        \T{\cmd{setTeam}\{\{Mega Team\} A\}}.
    \end{command}
    \end{frame}


\subsection{Die Daten anzeigen}

\begin{frame}[fragile,allowframebreaks]{Die Daten anzeigen}
    \begin{command}{typesetAuthors}{}
        Setzt die Autoren gemäß \jmark[\T{showmail}]{mrk:showmail} mit oder
        ohne E-Mail-Adresse, so ergibt \cmd{typesetAuthors}: \typesetAuthors.
        Bei mehreren werden Kommas und das \say{und} korrekt gesetzt.
    \end{command}

    \begin{command}{thesosversion}{}
        Gibt die aktuelle Version der \T{sopra-seraphim}-Dokumentklasse aus. \notetext{Hinweis: über \blatex{\\value\{sosversion\}} lässt sich
        die Version als $4$-stellige Nummer erhalten: \arabic{sosversion}.}
    \end{command}

    \begin{command}{DateConvert}{\manArg{YYYY-MM-DD}}
        Konvertiert ein Datum in eine schönere Form, so:
    \begin{plainlatex}[morekeywords={[5]{\\DateConvert}}]
\DateConvert{2019-12-02} % :yields: !*\solGet{comments}{\DateConvert{2019-12-02}}*!
    \end{plainlatex}
    \end{command}
\end{frame}

\subsection{Generelle Formatierungsbefehle}
\begin{frame}[fragile]{Generelle Formatierungsbefehle}
\begin{command}{imptext}{\manArg{Text}}
    Hebt \T{Text} als \imptext{sehr wichtig} hervor.
\end{command}

\begin{command}{notetext}{\manArg{Text}}
    Setzt \T{Text} als eine \notetext{beiläufige Notiz}.
\end{command}
\end{frame}

\subsection{Farben}
\begin{frame}[fragile]{Farben}
    \def\colshow#1{\T{#1} \tikz[baseline=-0.6ex]{\draw[fill=#1] circle (4pt);}}
    Die folgenden Farben stehen zur Verfügung: \foreach[count=\i] \col in {cprimary,cakzent,clight,cimportant,cwhite} {\ifnum\i>1,~\fi\colshow{\col}} und \colshow{chighlight}.
\end{frame}

\section{Verbundene Klassen und Pakete}
\begin{frame}[fragile]{Verbundene Klassen und Pakete}
Im Zuge dieser \LaTeXe-Dokumentklasse sind noch einige andere Pakete entstanden. Zum aktuellen Kompilierstand existiert (Pakete und Klassen, ohne Link sind dieser beigelegt und hier: \url{https://github.com/EagleoutIce/sopra-collection} zu finden):
\begin{multicols}{3}
    \begin{itemize}
        \foreach \i/\u in {eagle-maps/{https://github.com/EagleoutIce/eagle-maps},LILLY-Framework/{https://github.com/EagleoutIce/LILLY},sopra-documentation/,sopra-models/,sopra-listings/,sopra-requirements/,sopra-attachments/} {
            \item \T{\i}\expandafter\ifx\expandafter\\\u\\\else\footnote{\url{\u}}\fi
        }
    \end{itemize}
\end{multicols}
\end{frame}

\end{document}
