\documentclass{sopra-base} 

\usepackage{sopra-documentation}
\usepackage{sopra-changelog}

\title{Das 'sopra-changelog'-Paket}
\subtitle[Dokumentation für das 'sopra-changelog'-Paket]{Dokumentation für das 'sopra-changelog'-Paket~$\mid$~Version \thesocversion}

\duedate{2019-12-2}

\keywords{Dokumentation,sopra-changelog,sopra,uni ulm,uulm,Paket}
\authors{Florian Sihler (florian.sihler@uni-ulm.de)}

\group{Die Affenbande}

\setTargetVersion{5}

\begin{document}

    \maketitle%
%
    %\tableofcontents
%

%
%
% ....###....##.......##........######...########.##.....##.########.####.##....##.########..######.
% ...##.##...##.......##.......##....##..##.......###...###.##........##..###...##.##.......##....##
% ..##...##..##.......##.......##........##.......####.####.##........##..####..##.##.......##......
% .##.....##.##.......##.......##...####.######...##.###.##.######....##..##.##.##.######....######.
% .#########.##.......##.......##....##..##.......##.....##.##........##..##..####.##.............##
% .##.....##.##.......##.......##....##..##.......##.....##.##........##..##...###.##.......##....##
% .##.....##.########.########..######...########.##.....##.########.####.##....##.########..######.
%
%

\section{Allgemeines}
\subsection{Warum, wieso, weshalb?}
    Dieses \LaTeXe-Paket wurde im Rahmen des Sopras im 
    Wintersemester 2019 und Sommersemester 2020 verfasst und dient zur automatischen generierung von Changelogs für das \imptext{Teams 20}. Diese Dokumentation wurde zusammen mit der 
    \T{sopra-base.cls} sowie dem Paket \T{sopra-documentation.sty} kreiert.\par
    Zum Visualisieren der einzelnen Code-Ausschnitte wird das
    \T{sopra-listings}-Paket verwendet.
    Das zugehörige Paket sollte ebenfalls in dieses Dokument eingebettet sein: \scalebox{0.65}{\attachfile[subject={sopra-changelog.sty}]{sopra-changelog.sty}}. \notetext{Date Paket wurde binnen drei Stunden entwickelt und programmiert, es befindet sich aktuell in der Entwicklung.}
\subsection{Abhängigkeiten}
    Dieses Paket bindet die folgenden Paketen mit ein:
    \begin{multicols}{4}
        \begin{itemize}
            \def\pkgparse#1:#2\@nil{%
                \T{#1}\ifx\\#2\\\else\textsuperscript{(#2)}\fi%
            }
            \foreach \pkg in {environ:,fontawesome:,marginfix:,hyperref:} {
                \item \expandafter\pkgparse\pkg\@nil
            }
        \end{itemize}
    \end{multicols}
    All diese Pakete sollten Teil der gängigen \LaTeX-Distribution sein.

\subsection{Die Installation}
    Das Paket wird nicht als \T{.dtx} ausgeliefert, weswegen sich die 
    folgenden Möglichkeiten ergeben:
    \begin{itemize}
        \item Das Paket kann in dasselbe Verzeichnis wie das Dokument
                gesetzt werden. In diesem Fall lautet die Einbindungsanweisung:
\begin{plainlatex}
\usepackage{sopra-changelog}
\end{plainlatex}
        \item Das Paket kann in ein Unterverzeichnis/in ein mit
                dem Dokument ausgeliefertes Verzeichnis gelegt werden. In
                diesem Fall erfolgt die Angabe durch den (relativen-) Pfad:
\begin{plainlatex}
\usepackage{./Mein/Pfad/zu/sopra-changelog}
\end{plainlatex}
        \item Man kann das Paket (mittels eines Symlinks oder ähnlichem)
              in einen eigenen \emph{texmf}-Baum ablegen.
              So kann zum Beispiel auf Linux unter der Verwendung von texlive
              das Paket hier abgelegt werden: \bvoid{\~/texmf/tex/latex/}.
              Das Verzeichnis kann erstellt und anschließend mittels
              \bbash{texhash \~/texmf} aktualisiert werden. Nun kann
              das Paket wie jede andere installierte Paket verwendet werden:
\begin{plainlatex}
\usepackage{sopra-changelog}
\end{plainlatex}
    \end{itemize}

\subsection{Weitere Besonderheiten}
In Version \thesocversion{} (\cmdref{thesocversion}) gibt es keine weiteren
Besonderheiten.

%
%
% .##....##..#######..##....##.########.####..######...##.....##.########.....###....########.####..#######..##....##
% .##...##..##.....##.###...##.##........##..##....##..##.....##.##.....##...##.##......##.....##..##.....##.###...##
% .##..##...##.....##.####..##.##........##..##........##.....##.##.....##..##...##.....##.....##..##.....##.####..##
% .#####....##.....##.##.##.##.######....##..##...####.##.....##.########..##.....##....##.....##..##.....##.##.##.##
% .##..##...##.....##.##..####.##........##..##....##..##.....##.##...##...#########....##.....##..##.....##.##..####
% .##...##..##.....##.##...###.##........##..##....##..##.....##.##....##..##.....##....##.....##..##.....##.##...###
% .##....##..#######..##....##.##.......####..######....#######..##.....##.##.....##....##....####..#######..##....##
%
%


\section{Paket-Konfiguration}    
    \subsection{Akzeptierte Parameter}
    In Version \thesocversion{} werden keine Parameter akzeptiert.

%
%
% .########..########.########.########.##.....##.##.......########
% .##.....##.##.......##.......##.......##.....##.##.......##......
% .##.....##.##.......##.......##.......##.....##.##.......##......
% .########..######...######...######...#########.##.......######..
% .##.....##.##.......##.......##.......##.....##.##.......##......
% .##.....##.##.......##.......##.......##.....##.##.......##......
% .########..########.##.......########.##.....##.########.########
%
%

\section{Befehle- und Umgebungen}

Es gilt zu beachten, dass das Präfix \T{env@} nur auf die Natur einer Umgebung hinweist und nicht zum eigentlichen Bezeichner zuzuordnen ist!\par{}
Weiter gilt: Damit die Änderungen richtig aufgelöst werden können, muss das Dokument
in der Regel zwei mal kompiliert werden um eine korrekte Anzeige zu erzeugen.

\subsection{Versionsänderungen angeben}

\begin{environment}{version}{\optArg{changes}\manArg{Introduced Version}\manArg{Title}}
    Die Hauptumgebung des Paketes und wird für die Versionierung verwendet. Ein wichtiger Hinweis vorne weg: \imptext{Der Inhalt der Umgebung wird nur dann gesetzt, wenn \cmdref{setTargetVersion} mindestens auf \T{Introduced Version} gesetzt ist! Sonst gibt es die Änderung ja noch nicht.}\par{}
    Das Feld \T{Introduced Version} gibt an mit welcher Versionsnummer (von \(0\) an gezählt) die in der Umgebung eingeschlossenen Komponenten zur Verfügung stehen. Identifiziert werden diese durch den \emph{eindeutigen} Titel in \T{Title}.\par{}
    Das wichtigste Feld ist allerdings \T{changes}. Hierbei kann eine kommaseparierte Liste an Feldern übergeben die entweder der Syntax \say{\bvoid{changed in: :lan:VerID:ran: (:lan:Description:ran:)}} oder der Syntax \say{\bvoid{removed in: :lan:VerID:ran: (:lan:Description:ran:)}} gehorchen müssen. Ein Beispiel:
\begin{latex}
\begin{version}[%
    changed in: 2 (Added Some funny stuff),%
    changed in: 3 (Removed stupid sentences),%
    removed in: 4 (Rendered to be stupid!)
]{1}{Koordinaten des Spielfeldes}
    Hier kann nun alles stehen was man sich so vorstellen kann. Auch normaler \LaTeX-Code!
\end{version}
\end{latex}
    Dies erzeugt (Hinweis: Dieses Dokuent wurde mit \blatex{\\setTargetVersion\{5\}} kompiliert):\\
\begin{version}[%
    changed in: 2 (Added Some funny stuff),%
    changed in: 3 (Removed stupid sentences),%
    removed in: 4 (Rendered to be stupid!)
]{1}{Koordinaten des Spielfeldes}
Hier kann nun alles stehen was man sich so vorstellen kann. Auch normaler \LaTeX-Code!
\end{version}
\end{environment}

\begin{command}{simpleversion}{\optArg{changes}\manArg{Introduced Version}\manArg{Title}\manArg{Content}}
    Einfach nur ein kleiner Wrapper, der \envref{version} mit \T{Content} aufruft.
\end{command}

\begin{command}{setVersionData}{\manArg{Version ID}\manArg{Version Date}\manArg{Version Description}}
    Speichert für \cmdref{thechangelog} Metadaten zur Version mit Versionsnummer \T{Version ID}. Das \T{Version-Date} ist in \T{YYYY-MM-DD} anzugeben!
\end{command}

\subsection{Versionänderungen setzen}

\begin{command}{thechangelog}{}
    Setzt den Verlauf über alle Änderungen. Für die Gliederung wird \cmd{section} und \cmd{subsection} verwendet.
\end{command}

\subsection{Weitere Daten zuweisen}

\begin{command}{setTargetVersionDate}{\manArg{Date}}
    Bisher ohne Bedeutung!
\end{command}


\subsection{Hilfreiche Befehle}

\begin{command}{thesocversion}{}
    Gibt die aktuelle Version der \T{sopra-paper}-Dokumentklasse aus. \notetext{Hinweis: über \blatex{\\value\{socversion\}} lässt sich
    die Version als $4$-stellige Nummer erhalten: \arabic{socversion}.}
\end{command}

\end{document}