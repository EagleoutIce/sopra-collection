\documentclass{sopra-base}

\usepackage{sopra-documentation}
\usepackage{sopra-standard}

\title{Das 'sopra-standard'-Paket}
\subtitle[Dokumentation für das 'sopra-standard'-Paket]{Dokumentation für das 'sopra-standard'-Paket | Version \thesosversion}

\duedate{2019-12-2}

\keywords{Dokumentation,sopra-standard,sopra,uni ulm,uulm,Paket}
\authors{Florian Sihler (florian.sihler@uni-ulm.de)}

\group{Die Affenbande}



\begin{document}

    \maketitle

%
%
% ....###....##.......##........######...########.##.....##.########.####.##....##.########..######.
% ...##.##...##.......##.......##....##..##.......###...###.##........##..###...##.##.......##....##
% ..##...##..##.......##.......##........##.......####.####.##........##..####..##.##.......##......
% .##.....##.##.......##.......##...####.######...##.###.##.######....##..##.##.##.######....######.
% .#########.##.......##.......##....##..##.......##.....##.##........##..##..####.##.............##
% .##.....##.##.......##.......##....##..##.......##.....##.##........##..##...###.##.......##....##
% .##.....##.########.########..######...########.##.....##.########.####.##....##.########..######.
%
%

\section{Allgemeines}
\subsection{Warum, wieso, weshalb?}
    Dieses \LaTeXe-Paket wurde im Rahmen des Sopras im
    Wintersemester 2019 und Sommersemester 2020 verfasst und diente der Gestaltung des Standarddokuments. Diese Dokumentation wurde zusammen mit der
    \T{sopra-base.cls} sowie dem Paket \T{sopra-documentation.sty} kreiert.\par
    Zum Visualisieren der einzelnen Code-Ausschnitte wird das
    \T{sopra-listings}-Paket verwendet.
    Das zugehörige Paket sollte ebenfalls in dieses Dokument eingebettet sein: \scalebox{0.65}{\attachfile[subject={sopra-standard.sty}]{sopra-standard.sty}}.
\subsection{Abhängigkeiten}
    Dieses Paket bindet die folgenden Paketen mit ein:
    \begin{multicols}{4}
        \begin{itemize}
            \def\pkgparse#1:#2\@nil{%
                \T{#1}\ifx!#2!\else\textsuperscript{(#2)}\fi%
            }
            \foreach \pkg in {fontawesome:,xstring:,array:,sopra-changelog:} {
                \item \expandafter\pkgparse\pkg\@nil
            }
        \end{itemize}
    \end{multicols}
    All diese Pakete sollten Teil der gängigen \LaTeX-Distribution sein, beziehungsweise zusammen mit diesem Paket ausgeliefert werden.

\subsection{Die Installation}
    Das Paket wird nicht als \T{.dtx} ausgeliefert, weswegen sich die
    folgenden Möglichkeiten ergeben:
    \begin{itemize}
        \item Das Paket kann in dasselbe Verzeichnis wie das Dokument
                gesetzt werden. In diesem Fall lautet die Einbindungsanweisung:
\begin{plainlatex}
\usepackage{sopra-standard}
\end{plainlatex}
        \item Das Paket kann in ein Unterverzeichnis/in ein mit
                dem Dokument ausgeliefertes Verzeichnis gelegt werden. In
                diesem Fall erfolgt die Angabe durch den (relativen-) Pfad:
\begin{plainlatex}
\usepackage{./Mein/Pfad/zu/sopra-standard}
\end{plainlatex}
        \item Man kann das Paket (mittels eines Symlinks oder ähnlichem)
              in einen eigenen \emph{texmf}-Baum ablegen.
              So kann zum Beispiel auf Linux unter der Verwendung von texlive
              das Paket hier abgelegt werden: \bvoid{\~/texmf/tex/latex/}.
              Das Verzeichnis kann erstellt und anschließend mittels
              \bbash{texhash \~/texmf} aktualisiert werden. Nun kann
              das Paket wie jede andere installierte Paket verwendet werden:
\begin{plainlatex}
\usepackage{sopra-standard}
\end{plainlatex}
    \end{itemize}

\subsection{Weitere Besonderheiten}
In Version \thesosversion{} (\cmdref{thesosversion}) gibt es keine weiteren
Besonderheiten.

%
%
% .##....##..#######..##....##.########.####..######...##.....##.########.....###....########.####..#######..##....##
% .##...##..##.....##.###...##.##........##..##....##..##.....##.##.....##...##.##......##.....##..##.....##.###...##
% .##..##...##.....##.####..##.##........##..##........##.....##.##.....##..##...##.....##.....##..##.....##.####..##
% .#####....##.....##.##.##.##.######....##..##...####.##.....##.########..##.....##....##.....##..##.....##.##.##.##
% .##..##...##.....##.##..####.##........##..##....##..##.....##.##...##...#########....##.....##..##.....##.##..####
% .##...##..##.....##.##...###.##........##..##....##..##.....##.##....##..##.....##....##.....##..##.....##.##...###
% .##....##..#######..##....##.##.......####..######....#######..##.....##.##.....##....##....####..#######..##....##
%
%


\section{Paket-Konfiguration}
    \subsection{Akzeptierte Parameter}
    In Version \thesosversion{} werden keine Parameter akzeptiert.

%
%
% .########..########.########.########.##.....##.##.......########
% .##.....##.##.......##.......##.......##.....##.##.......##......
% .##.....##.##.......##.......##.......##.....##.##.......##......
% .########..######...######...######...#########.##.......######..
% .##.....##.##.......##.......##.......##.....##.##.......##......
% .##.....##.##.......##.......##.......##.....##.##.......##......
% .########..########.##.......########.##.....##.########.########
%
%

\section{Befehle- und Umgebungen}

Es gilt zu beachten, dass das Präfix \T{env@} nur auf die Natur einer Umgebung hinweist und nicht zum eigentlichen Bezeichner zuzuordnen ist!\par{}
Das Paket definiert zugrundeliegende Umgebungen die es zur stilisierung verwendet.
Diese werden in der Dokumentation nicht vorgestellt und sollten auch nicht direkt
verwendet werden.

\subsection{Die Umgebungen}
\begin{command}{CreateNewStandardType}{\manArg{settings}\manArg{environment name}\manArg{counter name}}
    Erstellt eine neue Umgebung mit dem Bezeichner \T{environment name}. Die Umgebung wird versioniert, die Schlüssel sind als kommaseparierte Liste aus \say{key: value}-Paaren zu übergeben. Als Beispiel dient hier die Definition
    des Kommentars:
\begin{plainlatex}[morekeywords={[5]{\\CreateNewStandardType,\\footnotesize,\\faCommentsO}}]
\CreateNewStandardType{symbol: \footnotesize!**!\faCommentsO,title: Kommentar}
    {comment}{ctrcomment}
\end{plainlatex}
Es sind die folgenden Einstellungen möglich: \begin{description}
    \item[author:] Setzt einen Autor, kann mehrfach angegeben werden um mehrere zu vermerken.
    \item[bookmark:] Setzt die bookmark-Tiefe bis zu der die Umgebung vermerkt werden soll.
    \item[symbol:] Setzt das Symbol, welches zum Text angezeigt wird.
    \item[title:] Setzt den Titel der Umgebung.
    \item[flags:] Erlaubt es Flaggen zusätzlich zur Box zur vermerken die Metadaten bezeichnen.
    \item[status:] Erlaubt es den Status für den Kommentar oder die Definition zu setzen (zum Beispiel: \T{OK}).
    \item[counter:] Setzt den Zähler, den es zu verwenden gilt.
\end{description}
Alle diese Einstellungen können auch manuell bei der Angabe der Umgebung überschrieben werden (dies empfiehlt sich zum Beispiel im Falle des Autoren-Tags).
\end{command}

Es werden die folgenden Umgebungen definiert: \env{filedef}, \env{comment}, \env{datadef}, \env{messagedef}, \env{cmdlinedef} und \env{begdef}.

\subsection{Querverweise im Code}

\begin{command}{CreateCodeHyperLink}{\optArg{Sequenz}\manArg{Ziel}}
    Benötigt \blatex{sopra-listings} und sorgt dafür, dass die \say{Sequenz} nun zu einem Hyperlink wird, welcher auf den durch \blatex{\\label} gesetzten Marker \say{\T{std:Ziel}} zeigt. Das \T{std:} wird hierbei \emph{nur genau dann} automatisch angehängt, wenn keine \say{explizite} Sequenz übergeben wird. In diesem Fall erwartet der Befehl eine Liste an Begriffen, die automatisch zu gleichnamigen Markern springen:
\begin{plainlatex}[morekeywords={[5]{\\CreateCodeHyperLink}}]
\CreateCodeHyperLink{CharacterDescription,GenderEnum,PropertyEnum,GadgetEnum,Gadget,NPCEnum,FieldStateEnum}
\end{plainlatex}
    Gegenüber:
\begin{plainlatex}[morekeywords={[5]{\\CreateCodeHyperLink}}]
\CreateCodeHyperLink[RequestItemChoiceMessage]{std:Nachricht: Charakter- und Gadgetangebot}
\CreateCodeHyperLink[ItemChoiceMessage]{std:Nachricht: Charakter- oder Gadgetwahl}
\end{plainlatex}
\end{command}

\begin{command}{setComment}{\manArg{Kommentar}}
    Setzt den Kommentar, der mit dem nächsten \cmdref{Chapter}-Befehl in den Table-Of-Contents gesetzt wird.
\end{command}

\begin{command}{Chapter}{\manArg{Bezeichner}}
    Arbeitet wie \cmd{chapter}, bindet aber \cmdref{setComment} mit ein:
\begin{plainlatex}[morekeywords={[5]{\\setComment,\\Chapter}}]
\setComment{Grundlegende Informationen, wie dieses Dokument zu lesen ist.}
\Chapter{Allgemeines}
\end{plainlatex}
\end{command}

\subsection{Allgemeine Befehle}

\begin{command}{thesosversion}{}
    Liefert die aktuelle Version des Pakets. So ergibt: \cmd{thesosversion}: \thesosversion\\
    \notetext{Hinweis: über \blatex{\\value\{sosversion\}} lässt sich
    die Version als $4$-stellige Nummer erhalten: \arabic{sosversion}.}
\end{command}

Weiter definiert werden noch einige Shortcuts wie \cmd{say} und \cmd{info}, die hier bald Dokumentiert werden sollen\ldots

\end{document}