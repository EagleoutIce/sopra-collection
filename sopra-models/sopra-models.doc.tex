\documentclass{sopra-base} 

\usepackage{sopra-documentation}
\usepackage{sopra-models}

\title{Das 'sopra-models'-Paket}
\subtitle[Dokumentation für das 'sopra-models'-Paket]{Dokumentation für das 'sopra-models'-Paket~$\mid$~Version \thesomversion}

\duedate{2019-12-2}

\keywords{Dokumentation,sopra-models,sopra,uni ulm,uulm,Paket}
\authors{Florian Sihler (florian.sihler@uni-ulm.de)}

\group{Die Affenbande}
\begin{document}
    \maketitle%
%
    %\tableofcontents
%

%
%
% ....###....##.......##........######...########.##.....##.########.####.##....##.########..######.
% ...##.##...##.......##.......##....##..##.......###...###.##........##..###...##.##.......##....##
% ..##...##..##.......##.......##........##.......####.####.##........##..####..##.##.......##......
% .##.....##.##.......##.......##...####.######...##.###.##.######....##..##.##.##.######....######.
% .#########.##.......##.......##....##..##.......##.....##.##........##..##..####.##.............##
% .##.....##.##.......##.......##....##..##.......##.....##.##........##..##...###.##.......##....##
% .##.....##.########.########..######...########.##.....##.########.####.##....##.########..######.
%
%

\section{Allgemeines}
\subsection{Warum, wieso, weshalb?}
    Dieses \LaTeXe-Paket wurde im Rahmen des Sopras im 
    Wintersemester 2019 und Sommersemester 2020 verfasst und dient als
    Grundlage für die Präsentation von (UML)-Diagrammen und Modellen
    des \imptext{Teams 20}. Diese Dokumentation wurde zusammen mit der 
    \T{sopra-base.cls} sowie dem Paket \T{sopra-documentation.sty} kreiert.\par
    \imptext{Wichtig: Ein großteil dieses Pakets basiert auf \T{tikz-uml} und 
    wird hier dokumentiert: \url{https://perso.ensta-paris.fr/~kielbasi/tikzuml/}}.\newline
    Zum Visualisieren der einzelnen Code-Ausschnitte wurde die 
    \T{LILLYxLISTINGS}-Bibliothek des Lilly-Frameworks\footnote{\url{https://github.com/EagleoutIce/LILLY}} von
    Florian Sihler verwendet. Es ist für die Verwendung des Pakets (\jmark[\T{nolistings}]{mrk:nolistings})
    nicht relevant.\par
    Das zugehörige Paket sollte ebenfalls in dieses Dokument eingebettet sein: \scalebox{0.65}{\attachfile[subject={sopra-models.sty}]{sopra-models.sty}}.
\subsection{Abhängigkeiten}
    Dieses Paket bindet die folgenden Paketen mit ein:
    \begin{multicols}{3}
        \begin{itemize}
            \def\pkgparse#1:#2\@nil{%
                \T{#1}\ifx\\#2\\\else\textsuperscript{(#2)}\fi%
            }
            \foreach \pkg in {tikz:,newfloat:,caption:,float:,ifthen:,xstring:,calc:,pgfopts:} {
                \item \expandafter\pkgparse\pkg\@nil
            }
        \end{itemize}
    \end{multicols}
    All diese Pakete sollten Teil der gängigen \LaTeX-Distribution sein. Weiter werden von Ti\textit{k}Z die folgenden
    Bibliotheken benutzt: \T{backgrounds}, \T{arrows}, \T{shapes}, \T{fit}, \T{shadows} und \T{decorations.markings}.

\subsection{Die Installation}
    Das Paket wird nicht als \T{.dtx} ausgeliefert, weswegen sich die 
    folgenden Möglichkeiten ergeben:
    \begin{itemize}
        \item Das Paket kann in dasselbe Verzeichnis wie das Dokument
                gesetzt werden. In diesem Fall lautet die Einbindungsanweisung:
\begin{plainlatex}
\usepackage{sopra-models}
\end{plainlatex}
        \item Das Paket kann in ein Unterverzeichnis/in ein mit
                dem Dokument ausgeliefertes Verzeichnis gelegt werden. In
                diesem Fall erfolgt die Angabe durch den (relativen-) Pfad:
\begin{plainlatex}
\usepackage{./Mein/Pfad/zu/sopra-models}
\end{plainlatex}
        \item Man kann das Paket (mittels eines Symlinks oder ähnlichem)
              in einen eigenen \emph{texmf}-Baum ablegen.
              So kann zum Beispiel auf Linux unter der Verwendung von texlive
              das Paket hier abgelegt werden: \bvoid{\~/texmf/tex/latex/}.
              Das Verzeichnis kann erstellt und anschließend mittels
              \bbash{texhash \~/texmf} aktualisiert werden. Nun kann
              das Paket wie jede andere installierte Paket verwendet werden:
\begin{plainlatex}
\usepackage{sopra-models}
\end{plainlatex}
    \end{itemize}

\subsection{Weitere Besonderheiten}
In Version \thesomversion{} (\cmdref{thesomversion}) gibt es keine weiteren
Besonderheiten.

%
%
% .##....##..#######..##....##.########.####..######...##.....##.########.....###....########.####..#######..##....##
% .##...##..##.....##.###...##.##........##..##....##..##.....##.##.....##...##.##......##.....##..##.....##.###...##
% .##..##...##.....##.####..##.##........##..##........##.....##.##.....##..##...##.....##.....##..##.....##.####..##
% .#####....##.....##.##.##.##.######....##..##...####.##.....##.########..##.....##....##.....##..##.....##.##.##.##
% .##..##...##.....##.##..####.##........##..##....##..##.....##.##...##...#########....##.....##..##.....##.##..####
% .##...##..##.....##.##...###.##........##..##....##..##.....##.##....##..##.....##....##.....##..##.....##.##...###
% .##....##..#######..##....##.##.......####..######....#######..##.....##.##.....##....##....####..#######..##....##
%
%


\section{Paket-Konfiguration}    
    \subsection{Akzeptierte Parameter}
    Dieses Paket akzeptiert selbst keine weiteren Pakete!

%
%
% .########..########.########.########.##.....##.##.......########
% .##.....##.##.......##.......##.......##.....##.##.......##......
% .##.....##.##.......##.......##.......##.....##.##.......##......
% .########..######...######...######...#########.##.......######..
% .##.....##.##.......##.......##.......##.....##.##.......##......
% .##.....##.##.......##.......##.......##.....##.##.......##......
% .########..########.##.......########.##.....##.########.########
%
%

\section{Befehle- und Umgebungen}

Es gilt zu beachten, dass das Präfix \T{env@} nur auf die Natur einer Umgebung hinweist und nicht zum eigentlichen Bezeichner zuzuordnen ist!


\subsection{Floats}

\begin{environment}{model}{\optArg{Placement}}
    Ein \T{float}, ganz analog zu \env{figure} und \env{table}.
\end{environment}

Beispiel:
\begin{plainlatex}[morekeywords={[3]{model}}]
    \begin{model}
        \centering
        Isch bin ein Modell.
        \caption[Und ich eigentlich kürzer.]{Ich bin der Titel.}
    \end{model}
\end{plainlatex}
Ergibt:
\begin{model}
    \centering
    Isch bin ein Modell.
    \caption[Und ich eigentlich kürzer.]{Ich bin der Titel.}
\end{model}

Diese lassen durch \cmd{listofmodel} auflisten:

\listofmodel

\end{document}