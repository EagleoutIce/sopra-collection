\documentclass{sopra-base} 

\usepackage{sopra-documentation}

\title{Das 'sopra-documentation'-Paket}
\subtitle[Dokumentation für das 'sopra-documentation'-Paket]{Dokumentation für das 'sopra-documentation'-Paket~$\mid$~Version \thesodversion}

\duedate{2019-12-2}

\keywords{Dokumentation,sopra-documentation,sopra,uni ulm,uulm,Paket}
\authors{Florian Sihler (florian.sihler@uni-ulm.de)}

\group{Die Affenbande}
\begin{document}
    \maketitle%
%
    %\tableofcontents
%

%
%
% ....###....##.......##........######...########.##.....##.########.####.##....##.########..######.
% ...##.##...##.......##.......##....##..##.......###...###.##........##..###...##.##.......##....##
% ..##...##..##.......##.......##........##.......####.####.##........##..####..##.##.......##......
% .##.....##.##.......##.......##...####.######...##.###.##.######....##..##.##.##.######....######.
% .#########.##.......##.......##....##..##.......##.....##.##........##..##..####.##.............##
% .##.....##.##.......##.......##....##..##.......##.....##.##........##..##...###.##.......##....##
% .##.....##.########.########..######...########.##.....##.########.####.##....##.########..######.
%
%

\section{Allgemeines}
\subsection{Warum, wieso, weshalb?}
    Dieses \LaTeXe-Paket wurde im Rahmen des Sopras im 
    Wintersemester 2019 und Sommersemester 2020 verfasst und dient als
    Grundlage für die Erstellung von Dokumentationen für die anderen Pakete und Klassen
    des \imptext{Teams 20}, welche zusammen mit der \T{sopra-base.cls} kreiert
    werden.\par
    Zum Visualisieren der einzelnen Code-Ausschnitte wird das
    \T{sopra-listings}-Paket verwendet. Es ist für die Verwendung des Pakets (\jmark[\T{nolistings}]{mrk:nolistings})
    nicht relevant.\par
    Das zugehörige Paket sollte ebenfalls in dieses Dokument eingebettet sein: \scalebox{0.65}{\attachfile[subject={sopra-documentation.sty}]{sopra-documentation.sty}}.
\subsection{Abhängigkeiten}
    Dieses Paket bindet die folgenden Paketen mit ein:
    \begin{multicols}{3}
        \begin{itemize}
            \def\pkgparse#1:#2\@nil{%
                \T{#1}\ifx\\#2\\\else\textsuperscript{(#2)}\fi%
            }
            \foreach \pkg in {xcolor:,attachfile2:,multicol:,sopra-listings:{nur \jmark[\T{listings}]{mrk:listings}},blindtext:{nur \jmark[\T{dummy}]{mrk:dummy}},
                hyperref:} {
                \item \expandafter\pkgparse\pkg\@nil
            }
        \end{itemize}
    \end{multicols}
    All diese Pakete sollten Teil der gängigen \LaTeX-Distribution/mit diesem ausgeliefert worden sein.

\subsection{Die Installation}
    Das Paket wird nicht als \T{.dtx} ausgeliefert, weswegen sich die 
    folgenden Möglichkeiten ergeben:
    \begin{itemize}
        \item Das Paket kann in dasselbe Verzeichnis wie das Dokument
                gesetzt werden. In diesem Fall lautet die Einbindungsanweisung:
\begin{plainlatex}
\usepackage{sopra-documentation}
\end{plainlatex}
        \item Das Paket kann in ein Unterverzeichnis/in ein mit
                dem Dokument ausgeliefertes Verzeichnis gelegt werden. In
                diesem Fall erfolgt die Angabe durch den (relativen-) Pfad:
\begin{plainlatex}
\usepackage{./Mein/Pfad/zu/sopra-documentation}
\end{plainlatex}
        \item Man kann das Paket (mittels eines Symlinks oder ähnlichem)
              in einen eigenen \emph{texmf}-Baum ablegen.
              So kann zum Beispiel auf Linux unter der Verwendung von texlive
              das Paket hier abgelegt werden: \bvoid{\~/texmf/tex/latex/}.
              Das Verzeichnis kann erstellt und anschließend mittels
              \bbash{texhash \~/texmf} aktualisiert werden. Nun kann
              das Paket wie jede andere installierte Paket verwendet werden:
\begin{plainlatex}
\usepackage{sopra-documentation}
\end{plainlatex}
    \end{itemize}
    \subsection{Weitere Besonderheiten}
    \paragraph{v1.0.0:}
    In dieser Version gibt es keine weiteren Besonderheiten.
    \paragraph{\protect\thesodversion:}
    In dieser Version wurde \T{LILLYxLISTINGS} durch \T{sopra-listings} ersetzt!

%
%
% .##....##..#######..##....##.########.####..######...##.....##.########.....###....########.####..#######..##....##
% .##...##..##.....##.###...##.##........##..##....##..##.....##.##.....##...##.##......##.....##..##.....##.###...##
% .##..##...##.....##.####..##.##........##..##........##.....##.##.....##..##...##.....##.....##..##.....##.####..##
% .#####....##.....##.##.##.##.######....##..##...####.##.....##.########..##.....##....##.....##..##.....##.##.##.##
% .##..##...##.....##.##..####.##........##..##....##..##.....##.##...##...#########....##.....##..##.....##.##..####
% .##...##..##.....##.##...###.##........##..##....##..##.....##.##....##..##.....##....##.....##..##.....##.##...###
% .##....##..#######..##....##.##.......####..######....#######..##.....##.##.....##....##....####..#######..##....##
%
%


\section{Paket-Konfiguration}    
    \subsection{Akzeptierte Parameter}
    Das Paket akzeptiert, so wie die meisten, Argumente. 
    Bei Argumenten mit einer \say{Counter}-Option wird das jeweils standardmäßig aktive zuerst und das andere in Klammern
    geschrieben. So wird implizit:
\begin{plainlatex}
    \usepackage[nodebug,dummy,listings]{sopra-documentation}
\end{plainlatex}
    aufgerufen. Während wir mit:
\begin{plainlatex}
    \usepackage[nolistings]{sopra-documentation}
\end{plainlatex}
    das Dokument ohne das \T{LILLYxLISTINGS}-Paket kompilieren.

    \begin{argument}{nodebug}{debug}
        \label{mrk:debug}Im \T{debug}-Modus wird der Log, dank \cmd{errorcontextlines} ausführlicher gefasst (im Falle eines Fehlers).
    \end{argument}

    \begin{argument}{dummy}{nodummy}
        \label{mrk:nodummy}Im \T{dummy}-Modus wird das Paket \T{blindtext} eingebunden, sonst nicht.
    \end{argument}

    \begin{argument}{listings}{nolistings}
        \label{mrk:nolistings}Im \T{listings}-Modus wird das Paket \T{LILLYxLISTINGS} eingebunden, sonst nicht.
    \end{argument}

%
%
% .########..########.########.########.##.....##.##.......########
% .##.....##.##.......##.......##.......##.....##.##.......##......
% .##.....##.##.......##.......##.......##.....##.##.......##......
% .########..######...######...######...#########.##.......######..
% .##.....##.##.......##.......##.......##.....##.##.......##......
% .##.....##.##.......##.......##.......##.....##.##.......##......
% .########..########.##.......########.##.....##.########.########
%
%

\section{Befehle- und Umgebungen}

Es gilt zu beachten, dass das Präfix \T{env@} nur auf die Natur einer Umgebung hinweist und nicht zum eigentlichen Bezeichner zuzuordnen ist!


\subsection{Kompatibilitätsmodus}

\begin{center}
    \imptext{Dieser Modus steht in \thesodversion{} nicht zur Verfügung!}
\end{center}

Diese Befehle werden nur mit \jmark[\T{nolistings}]{mrk:nolistings} definiert, da sie sonst von \T{LILLYxLISTINGS} eingebunden
werden würden:

\begin{command}{T}{\manArg{Text}}
    Wird hier als Alias für \cmd{texttt} umgesetzt; setzt \T{Text} also in Schreibmaschinenschrift.
\end{command}


\begin{command}{lstcomment}{\manArg{Text}}
    Analog existieren \cmd{lstkwone}, \cmd{lstkwtwo}, \cmd{lstkwthree}, \cmd{lstkwfour}, \cmd{lstkwfive}, \cmd{lstkwsix}, \cmd{lststring} und \cmd{lstnumber} die \T{Text} farblich hervorheben.
\end{command}


\subsection{Generelle Befehle}

\begin{command}{email}{\manArg{E-Mail}}
    Setzt \T{E-Mail} als klickbare E-Mail-Adresse.
\end{command}

\begin{command}{thesodversion}{}
    Liefert die aktuelle Version des Pakets. So ergibt: \cmd{thesodversion}: \thesodversion\\
    \notetext{Hinweis: über \blatex{\\value\{sodversion\}} lässt sich
    die Version als $4$-stellige Nummer erhalten: \arabic{sodversion}.}
\end{command}

\begin{command}{manArg}{\manArg{Text}}
    Erlaubt es in \envref{command} oder vergleichbaren Umgebungen ein verpflichtendes Argument anzugeben. \notetext{So ist \T{Text} in der Befehlsdefinition mittels \cmd{manArg} gesetzt!}
\end{command}

\begin{command}{optArg}{\manArg{Text}}
    Erlaubt es in \envref{command} oder vergleichbaren Umgebungen ein optionales Argument anzugeben.
\end{command}

\begin{command}{cmd}{\manArg{Name}}
    Setzt \T{Name} als Befehl: \T{\cmd{cmd}\{Hi\}}: \cmd{Hi}.
\end{command}

\begin{command}{env}{\manArg{Name}}
    Setzt \T{Name} als Umgebung: \T{\cmd{env}\{Hi\}}: \env{Hi}.
\end{command}

\begin{command}{cmdref}{\manArg{Name}}
    Analog zu \cmdref{cmd}, allerdings wird (so wie genau hier auch) ein Link zur Erklärung von \T{Name} erstellt.
\end{command}

\begin{command}{envref}{\manArg{Name}}
    Analog zu \cmdref{env}, allerdings wird ein Link zur Erklärung von \T{Name} erstellt.
\end{command}

\begin{command}{argref}{\manArg{Name}\manArg{Counter}}
    Setzt ein Argument mit entsprechender Counter-Option.
\end{command}


\begin{environment}{command}{\manArg{Name}\manArg{Argumente}}
    Startet die Erklärung eines neuen Befehls mit Bezeichner \T{Name} (für \cmdref{cmdref}) und den Argumenten \T{Argumente}.
\end{environment}

\begin{environment}{environment}{\manArg{Name}\manArg{Argumente}}
    Startet die Erklärung einer neuen Umgebung mit Bezeichner \T{Name} (für \cmdref{envref}) und den Argumenten \T{Argumente} (so wie hier).
\end{environment}

\begin{environment}{argument}{\manArg{Name}\manArg{Conter-Name}}
    Startet die Erklärung eins neuen Arguments für das Paket/die Klase mit Bezeichner \T{Name} und dem Konter-Bezeichner \T{Conter-Name}.
\end{environment}

\begin{command}{say}{\manArg{Text}}
    Setzt \T{Text} in \say{Anführungszeichen}.
\end{command}

\begin{command}{jmark}{\optArg{Text}\manArg{Linkziel}}
    Funktioniert analog zum Pendant in \T{Lilly}, weil ich es so lieber mag und aus Reflex schreibe :D.
\end{command}

\subsection{Weitere Befehle}

Diese Befehle entstammen den eingebundenen Paketen und werden lediglich konfiguriert.

\begin{command}{attachfile}{\optArg{Options}\manArg{Pfad}}
    Bindet die Datei in \T{Pfad} ein. Als \T{Options} bietet sich zumindest \T{subject} an. So wird in dieser Dokumentation zum Beispiel das Paket mit folgender Zeile eingebunden:
\begin{plainlatex}
\attachfile[subject={sopra-documentation.sty}]{sopra-documentation.sty}
\end{plainlatex}
\end{command}

\begin{command}{textattachfile}{\optArg{Options}\manArg{Pfad}\manArg{Text}}
    Bindet die Datei in \T{Pfad} ein, macht aber anstelle eines Symbols \T{Text} zum klickbaren Link zum öffnen. Die Optionen sind analog zu \cmdref{attachfile}.
\end{command}

Für alle anderen verwendeten Methoden emfpiehlt es sich, den Quellcode selbst oder den Quellcode der Dokumentation näher zu betrachten.

\end{document}